\documentclass{article}

% Umlaut-Support
\usepackage[ngerman]{babel}
\usepackage[utf8]{inputenc}
\usepackage[T1]{fontenc}

% AMS
\usepackage{amsmath}
\usepackage{amssymb}
\usepackage{amsthm}

% Befehle
\newcommand{\m}[1]{\begin{pmatrix}#1\end{pmatrix}}

% Metadaten
\title{Verschiebung von Koordinatensystemen}
\author{Justus Seeck}
\date{\today}

\begin{document}
    \maketitle

    \tableofcontents

    \newpage

    \section{Einführung}

    \subsection{Vorwort}

    In der folgenden Arbeit wird lediglich die Verschiebung von Koordinatensystemen in der Ebene (also in zwei Dimensionen) behandelt.
    Die Methodik lässt sich jedoch auch auf die dritte Dimension übertragen. Auch die Anzahl der Verschiebungen ist beliebig, da man die
    hier verwendete Formel beliebig oft hintereinander anwenden kann, wird hier jedoch auf eine Verschiebung beschränkt.

    \subsection{Schreibweisen und Definitionen}
    
    \paragraph*{Punkte}

    Ein Punkt $P$ wurde bisher in der Form $P(x,y)$ dargestellt. Dies wird in der folgenden Arbeit nicht mehr verwendet. Stattdessen wird
    der Punkt $P$ in der Form $\m{x \\ y}$ dargestellt.

    
    \section{Vektoren}


\end{document}


Notizen und Hilfen

% Der Satz des Pytagoras lautet: $a^2 + b^2 = c^2$.
% Die PQ-Formel latuet:
    
% \[
%     x = - \frac{p}{2} \pm \sqrt{ \left ( \frac{p}{2}^2 \right ) - q }
% \]

% \[
%     \begin{pmatrix}
%         1 & 2 & 3 \\
%         4 & 5 & 6 \\
%         7 & 8 & 9

%     \end{pmatrix}
% \]