\documentclass{article}

% Umlaut-Support
\usepackage[ngerman]{babel}
\usepackage[utf8]{inputenc}
\usepackage[T1]{fontenc}

% AMS
\usepackage{amsmath}
\usepackage{amssymb}
\usepackage{amsthm}

% Koordinatensysteme
\usepackage{tkz-euclide}

% Befehle
\newcommand{\m}[1]{\begin{pmatrix}#1\end{pmatrix}}

\newcommand{\cspm}[1]{\begin{figure}[h]
    \begin{tikzpicture}
       \tkzInit[xmax=10,ymax=10,xmin=-10,ymin=-10]

        % Gitter
        \draw[step=5mm, gray, thin] (-5,-5) grid (5,5);

        % Koordinatensystem
        \foreach \x in {0,1,2,3,4,5}
            \draw (\x cm,1pt) -- (\x cm,-1pt) node[anchor=north] {$\x$};
        \foreach \y in {0,1,2,3,4,5}
            \draw (1pt,\y cm) -- (-1pt,\y cm) node[anchor=east] {$\y$};

        \foreach \x in {-1,-2,-3,-4,-5}
            \draw (\x cm,1pt) -- (\x cm,-1pt) node[anchor=north] {$\x$};
        \foreach \y in {-1,-2,-3,-4,-5}
            \draw (1pt,\y cm) -- (-1pt,\y cm) node[anchor=east] {$\y$};

        % Punkte und Vektoren
        #1
        
    \end{tikzpicture}
\end{figure}}

\newcommand{\csp}[1]{\begin{figure}[h]
    \begin{tikzpicture}
       \tkzInit[xmax=10,ymax=10,xmin=-10,ymin=-10]

        % Gitter
        \draw[step=5mm, gray, thin] (0,0) grid (10,10);

        % Koordinatensystem
        \foreach \x in {0,1,2,3,4,5,6,7,8,9,10}
            \draw (\x cm,1pt) -- (\x cm,-1pt) node[anchor=north] {$\x$};
        \foreach \y in {0,1,2,3,4,5,6,7,8,9,10}
            \draw (1pt,\y cm) -- (-1pt,\y cm) node[anchor=east] {$\y$};

        % Punkte und Vektoren
        #1
        
    \end{tikzpicture}
\end{figure}}

% Einstellungen
\setlength{\parindent}{0pt}

% Metadaten
\title{Verschiebung von Koordinatensystemen}
\author{Justus Seeck}
\date{\today}

\begin{document}
    \maketitle

    \tableofcontents

    \newpage

    \section{Einführung}

    \subsection{Vorwort}

    In der folgenden Arbeit wird lediglich die Verschiebung von Koordinatensystemen in der Ebene (also in zwei Dimensionen) behandelt.
    Die Methodik lässt sich jedoch auch auf die dritte Dimension übertragen. Auch die Anzahl der Verschiebungen ist beliebig, da man die
    hier verwendete Formel beliebig oft hintereinander anwenden kann, wird hier jedoch auf eine Verschiebung beschränkt.

    \subsection{Schreibweisen und Definitionen}
    
    \paragraph*{Punkte}

    Ein Punkt $P$ wurde bisher in der Form $P(x,y)$ dargestellt. Dies wird in der folgenden Arbeit nicht mehr verwendet. Stattdessen wird
    der Punkt $P$ in der Form $\m{x \\ y}$ dargestellt. Dies ist eine sogenannte Matrix mit zwei Zeilen und einer Spalte.
    Die erste Zeile enthält die x-Koordinate, die zweite Zeile die y-Koordinate.

    
    \section{Vektoren}

    Geometrische Vektoren sind Pfeile (oder Pfeilklassen). $\vec{v}$ ist ein Vektor.
    Vektoren werden durch ihre Länge in $x$-Richtung und $y$-Richtung definiert
    (bei drei Dimensionen zusätzlich in $z$-Richtung) und mit einem Pfeil über dem Buchstaben gekennzeichnet.
    Vektoren haben keine Feste Postition im Koordinatensystem. Sie können beliebig verschoben werden.
    Solange der Vektor alle seine Eigenschaften (Länge in $x$-, $y$- und ggf. $z$-Richtung) behält,
    ist er der gleiche Vektor. \textbf{Anhang: Vektoren 1}

    % Anhänge

    \newpage

    \section{Anhänge}

    \subsection{Vektoren 1}
    \cspm{
        % Vektor 1
        \draw[->](-4,0.5) -- (-2,1.5) node[above] {$\vec{v}$};

        % Vektor 2
        \draw[->](2,2) -- (4,3) node[above] {$\vec{v}$};

        % Vektor 3
        \draw[->](-1,-3) -- (1,-2) node[above] {$\vec{v}$};
    }
    Koordinatensystem mit dem Vektor $\vec{v} = \m{2 \\ 1}$ in drei verschiedenen Positionen.

\end{document}


% Notizen und Hilfen

% Der Satz des Pytagoras lautet: $a^2 + b^2 = c^2$.
% Die PQ-Formel latuet:
    
% \[
%     x = - \frac{p}{2} \pm \sqrt{ \left ( \frac{p}{2}^2 \right ) - q }
% \]

% \[
%     \begin{pmatrix}
%         1 & 2 & 3 \\
%         4 & 5 & 6 \\
%         7 & 8 & 9

%     \end{pmatrix}
% \]